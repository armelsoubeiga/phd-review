% Options for packages loaded elsewhere
\PassOptionsToPackage{unicode}{hyperref}
\PassOptionsToPackage{hyphens}{url}
%
\documentclass[
]{book}
\usepackage{amsmath,amssymb}
\usepackage{lmodern}
\usepackage{iftex}
\ifPDFTeX
  \usepackage[T1]{fontenc}
  \usepackage[utf8]{inputenc}
  \usepackage{textcomp} % provide euro and other symbols
\else % if luatex or xetex
  \usepackage{unicode-math}
  \defaultfontfeatures{Scale=MatchLowercase}
  \defaultfontfeatures[\rmfamily]{Ligatures=TeX,Scale=1}
\fi
% Use upquote if available, for straight quotes in verbatim environments
\IfFileExists{upquote.sty}{\usepackage{upquote}}{}
\IfFileExists{microtype.sty}{% use microtype if available
  \usepackage[]{microtype}
  \UseMicrotypeSet[protrusion]{basicmath} % disable protrusion for tt fonts
}{}
\makeatletter
\@ifundefined{KOMAClassName}{% if non-KOMA class
  \IfFileExists{parskip.sty}{%
    \usepackage{parskip}
  }{% else
    \setlength{\parindent}{0pt}
    \setlength{\parskip}{6pt plus 2pt minus 1pt}}
}{% if KOMA class
  \KOMAoptions{parskip=half}}
\makeatother
\usepackage{xcolor}
\IfFileExists{xurl.sty}{\usepackage{xurl}}{} % add URL line breaks if available
\IfFileExists{bookmark.sty}{\usepackage{bookmark}}{\usepackage{hyperref}}
\hypersetup{
  pdftitle={Review - Unsupervised classification for longitudinal data and trajectorie analysis},
  pdfauthor={Armel SOUBEIGA},
  hidelinks,
  pdfcreator={LaTeX via pandoc}}
\urlstyle{same} % disable monospaced font for URLs
\usepackage{longtable,booktabs,array}
\usepackage{calc} % for calculating minipage widths
% Correct order of tables after \paragraph or \subparagraph
\usepackage{etoolbox}
\makeatletter
\patchcmd\longtable{\par}{\if@noskipsec\mbox{}\fi\par}{}{}
\makeatother
% Allow footnotes in longtable head/foot
\IfFileExists{footnotehyper.sty}{\usepackage{footnotehyper}}{\usepackage{footnote}}
\makesavenoteenv{longtable}
\usepackage{graphicx}
\makeatletter
\def\maxwidth{\ifdim\Gin@nat@width>\linewidth\linewidth\else\Gin@nat@width\fi}
\def\maxheight{\ifdim\Gin@nat@height>\textheight\textheight\else\Gin@nat@height\fi}
\makeatother
% Scale images if necessary, so that they will not overflow the page
% margins by default, and it is still possible to overwrite the defaults
% using explicit options in \includegraphics[width, height, ...]{}
\setkeys{Gin}{width=\maxwidth,height=\maxheight,keepaspectratio}
% Set default figure placement to htbp
\makeatletter
\def\fps@figure{htbp}
\makeatother
\setlength{\emergencystretch}{3em} % prevent overfull lines
\providecommand{\tightlist}{%
  \setlength{\itemsep}{0pt}\setlength{\parskip}{0pt}}
\setcounter{secnumdepth}{5}
\usepackage{booktabs}
\usepackage{booktabs}
\usepackage{longtable}
\usepackage{array}
\usepackage{multirow}
\usepackage{wrapfig}
\usepackage{float}
\usepackage{colortbl}
\usepackage{pdflscape}
\usepackage{tabu}
\usepackage{threeparttable}
\usepackage{threeparttablex}
\usepackage[normalem]{ulem}
\usepackage{makecell}
\usepackage{xcolor}
\ifLuaTeX
  \usepackage{selnolig}  % disable illegal ligatures
\fi
\usepackage[]{natbib}
\bibliographystyle{plainnat}

\title{Review - Unsupervised classification for longitudinal data and trajectorie analysis}
\author{Armel SOUBEIGA}
\date{2022-04-20}

\begin{document}
\maketitle

{
\setcounter{tocdepth}{1}
\tableofcontents
}
\hypertarget{about}{%
\chapter*{About}\label{about}}
\addcontentsline{toc}{chapter}{About}

This paper reports on the bibliographic research of my thesis on unsupervised classification for longitudinal data and trajectories analysis.

\hypertarget{objectives}{%
\section{Objectives}\label{objectives}}

The objective is to synthesise the current state of research related to the topic (what others have done - said - found).

\hypertarget{definition}{%
\section{Definition}\label{definition}}

unsupervised classification : \textbf{Unsupervised classification refers to a set of methods whose objective is to establish or recover an existing typology characterising a set of n observations, from p characteristics measured on each of the observations.}

longitudinal data : \textbf{Longitudinal data is a series of repetitive observations of the same topics, collected over a period of time. Longitudinal data varies from cross-sectional data, since it tracks the same subjects over a period of time, whereas cross-sectional data samples different subjects at every point of time.}

trajectories data :

\hypertarget{delimitations}{%
\section{Delimitations}\label{delimitations}}

We are only interested in unsupervised classification as an analysis method, with this differents terms :

\begin{itemize}
\item
  unsupervised classification
\item
  unsupervised learning
\item
  clustering / partitioning
\end{itemize}

\hypertarget{keywords-definition}{%
\section{Keywords definition}\label{keywords-definition}}

Direct terms :

\begin{itemize}
\item
  unsupervised classification of trajectories
\item
  data unsupervised classification of longitudinal data
\end{itemize}

Synonymous terms :

\begin{itemize}
\item
  clustering of trajectories
\item
  automatical classification of trajectories
\item
  clustering of longitudinal data
\item
  automatical classification of longitudinal
\end{itemize}

Indirect terms :

\begin{itemize}
\item
  classification of \emph{life}, \emph{health}, \emph{care} pathways
\item
  analysis of longitudinal data
\item
  analysis of trajectories
\end{itemize}

\hypertarget{types-of-information}{%
\section{Types of information}\label{types-of-information}}

In the literature review, we are looking for theoretical analyses or examples of applications (longitudinal data, trajectory data, source code)

\hypertarget{the-ressources}{%
\section{The ressources}\label{the-ressources}}

\begin{table}
\centering\begingroup\fontsize{10}{12}\selectfont

\begin{tabular}{ll}
\toprule
Ressource & Type\\
\midrule
{}[Google Scholar](https://scholar.google.com/ "Google Scholar") & Articles académiques\\
{}[Cairn](https://www.cairn.info/) & Articles académiques\\
{}[Science Direct](https://www.sciencedirect.com/) & Articles académiques, Actes de congrès, Résumés de livres\\
\bottomrule
\end{tabular}
\endgroup{}
\end{table}

\hypertarget{bibliographic-review-management}{%
\section{Bibliographic review management}\label{bibliographic-review-management}}

\begin{enumerate}
\def\labelenumi{\arabic{enumi}.}
\tightlist
\item
  Do a quick refresh of the list of unsupervised and semi-supervised classification methods.
\item
  Find articles treating exactly with the subject (reference articles)
\item
  Extend the reading
\end{enumerate}

\begin{itemize}
\item
  For each article reading, I will summarize it according to : the title, authors,
  date and location of article publication, the problem, limits and advantages of the results.
\item
  Then, I will archive it to my account \textbf{\href{https://www.mendeley.com/reference-management/reference-manager}{mendeley}}
\item
  At end, I will do repport (presentation) to Violaine and the team if necessary.
\end{itemize}

\hypertarget{unsupervised-and-semi-supervised-classification-methods}{%
\chapter{unsupervised and semi-supervised classification methods}\label{unsupervised-and-semi-supervised-classification-methods}}

\hypertarget{lecture-1}{%
\section{Lecture 1}\label{lecture-1}}

Grira N, Crucianu M, \& Boujemaa N. Unsupervised and semi-supervised clustering:
A brief survey. In A review of machine learning techniques for processing multimedia
content, report of the muscle European network of excellence, 2004.

\url{http://cedric.cnam.fr/~crucianm/src/BriefSurveyClustering.pdf}

\hypertarget{lecture-2}{%
\section{Lecture 2}\label{lecture-2}}

E. Lebarbier, T. Mary-Huard. Classification non supervisée. AgroParisTech

\hypertarget{lecture-3}{%
\section{Lecture 3}\label{lecture-3}}

DJIBEROU MAHAMADOU Abdoul Jalil. Development of clustering algorithms for categorical data and applications
in Health

\url{https://perso.isima.fr/~viantoin/PAPERS/thesis/these-2021-09-DjiberouAbdoul.pdf}

\hypertarget{unsupervised-classification-for-longitudinal-data-and-trajectories-reference-articles}{%
\chapter{Unsupervised classification for longitudinal data and trajectories (reference articles)}\label{unsupervised-classification-for-longitudinal-data-and-trajectories-reference-articles}}

\hypertarget{paper-1}{%
\section{Paper 1}\label{paper-1}}

Philippe Apparicio, Mylène Riva and Anne-Marie Séguin. A comparison of two
methods for classifying trajectories: a case study on neighborhood poverty at the intra-metropolitan level in Montreal. European Journal of Geography, Space, Society, Territory, document 727, Online since 04 June 2015, connection on 07 July 2021.
URL: \url{http://journals.openedition.org/cybergeo/27035}

\hypertarget{paper-2}{%
\section{Paper 2}\label{paper-2}}

Classification et Prévision des Données Hétérogènes : Application aux Trajectoires et Séjours Hospitaliers.

\url{https://perso.univ-lyon1.fr/haytham.elghazel/Papers/Pdf/Rapport_These.pdf}

\hypertarget{extend-the-reading}{%
\chapter{Extend the reading}\label{extend-the-reading}}

  \bibliography{book.bib,packages.bib}

\end{document}
